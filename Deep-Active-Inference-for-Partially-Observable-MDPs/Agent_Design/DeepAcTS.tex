\documentclass{article}

\usepackage[margin=1in]{geometry}
\usepackage{amsfonts}
\usepackage{amsmath}
\usepackage[backend=biber]{biblatex}
\addbibresource{references.bib}

\title{DeepAcTS Agent Structure}

\begin{document}

\maketitle

\section{Notational Conventions}
I use capital letters like $P$  and $Q$ to denote an arbitrary probability distribution, as opposed to a \textit{particular} probability distribution like the Gaussian: $\mathcal{N}$ or Categorical distribution: $\mathcal{C}$. 

\

Lowercase letters that prefix a pair of parenthesis \textemdash \ like $p(\hdots)$ \textemdash \ are always used to denote a \textit{neural network}. This neural network either predicts the sufficient statistics for a multivariate Gaussian, or it predicts a categorical distribution. I hope it is sufficiently clear from the context as to which meaning is employed in particular instances. 

\

I use Greek letter sub/superscripts to label neural networks, their predicted values, the associated distribution that their predictions parameterize and any value sampled from this distribution. Hence $p_{\alpha}$ denotes a \textit{particular} neural network and $(\mu_{\alpha}, \Sigma_{\alpha})$ denote the network's predicted values. Finally, $\mathcal{N}_{\alpha}$ denotes the \textit{actual} approximate distribution, parameterized by $(\mu_{\alpha}, \Sigma_{\alpha})$.

\

If $x$ is a variable, then I use $\hat{x}^{(\alpha)}$ to denote that this variable has been \textit{sampled} from the probability distribution: $P_{\alpha}$. A subscript $\tau$ indexes a particular value in some time series. Hence, $\hat{x}_{\tau}^{(\alpha)}$ is the value sampled from $P_{\alpha}$ at time $\tau$. I use $\hat{x}^{(*)}$ to denote that $x$ is sampled from \textit{differing} sources, depending on the training/testing context. See Sections \ref{sec:training_phase} and \ref{sec:testing_phase} for details on where these values are sampled.

\

I use $\mathbb{A}$ to denote the \textit{action space}. This is the set of all action affordances for the agent at all times. Following \cite{DEEP-AIF-For-POMDPs}, I use $\tilde{G}_{\tau}$ to denote the \textit{Expected Free Energy} of each action in the action space, at time $\tau$. Hence the expected free energy for action $a_{\tau}$ is denoted $\tilde{g}(a_{\tau})$. 

\

Finally, $o_{\tau}$ denotes an \textit{observation} at time $\tau$. $s_{\tau}$ denotes the \textit{hidden state} at time $\tau$ and $a_{\tau}$ denotes the \textit{action} at time $\tau$.

\section{Models and VFE Loss}
I use the form of the Variational Free Energy from \cite{DEEP-AIF-For-POMDPs} as the free energy loss function. This is:

\begin{equation}
    \begin{aligned}
        \label{eq:true_VFE}
        -F_{\tau} &= -\mathbb{E}_{Q(s_{\tau})}[\log{(P(o_{\tau} \ | \ s_{\tau}))}] \\
        &+ \text{D}_{\text{KL}}[Q(s_{\tau}) \ \| \ P(s_{\tau} \ | \ s_{\tau - 1}, a_{\tau - 1})] \\
        &+ \text{D}_{\text{KL}}[Q(a_{\tau} \ | \ s_{\tau}) \ \| \ P(a_{\tau} \ | \ s_{\tau})]
    \end{aligned}
\end{equation}

$P(s_{\tau} \ | \ s_{\tau - 1}, a_{\tau - 1})$ is the prior state-transition probability. $Q(s_{\tau})$ is the \textit{Variational} posterior over hidden states. $P(o_{\tau} \ | \ s_{\tau})$ is the observation likelihood. The Variational action posterior is $Q(a_{\tau} \ | \ s_{\tau})$ and the \textit{true} action posterior is $P(a_{\tau} \ | \ s_{\tau})$.

\section{Neural Network Approximations}
Following a standard assumption in the literature, I aim to approximate $P(s_{\tau} \ | \ s_{\tau - 1}, a_{\tau - 1})$, $Q(s_{\tau})$ and $P(o_{\tau} \ | \ s_{\tau})$ as diagonal, multivariate Gaussian distributions. Hence, my neural network approximations to these densities output a mean vector: $\mu$ and a vector that constitutes the \textit{diagonal} entries of the covariance matrix: $\Sigma$. For the sake of numerical stability, my Gaussian models actually output \textit{log-variances}: $\log{(\Sigma)}$.

Hence any one of my Gaussian models may be notated as:

\begin{equation}
    \label{eq:Gauss_model_form}
    (\mu_{\alpha}(v), \log{(\Sigma_{\alpha}(v))}) = r_{\alpha}(v \ | \ \Omega)
\end{equation}

Here $\alpha$ denotes the parameters of the Gaussian neural network: $r_{\alpha}$. The variable $v$ is the variable over which the multivariate Gaussian predicted by $r_{\alpha}$ is defined. $v$ could be a state, observation or action (for example). $\Omega$ is an arbitrary set of variables upon which $v$ is conditioned. Hence, in equation \ref{eq:q_theta} we have that:

\begin{equation}
    \begin{aligned}
        \label{eq:gauss_model_example}
        \alpha &= \theta \\
        r_{\alpha} &= q_{\theta} \\
        v &= s_{\tau} \\
        \Omega &= \{\hat{s}_{\tau - 1}^{(\theta)}, \hat{a}_{\tau - 1}^{(*)}, \hat{o}_{\tau}^{(*)}\}
    \end{aligned}
\end{equation}

Thus are my approximations:

\begin{equation}
    \label{eq:q_theta}
    (\mu_{\theta}(s_{\tau}), \log{(\Sigma_{\theta}(s_{\tau}))}) = q_{\theta}(s_{\tau} \ | \ \hat{s}_{\tau - 1}^{(\theta)}, \hat{a}_{\tau - 1}^{(*)}, \hat{o}_{\tau}^{(*)})
\end{equation}

\begin{equation}
    \label{eq:p_phi}
    (\mu_{\phi}(s_{\tau}), \log{(\Sigma_{\phi}(s_{\tau}))}) = p_{\phi}(s_{\tau} \ | \ \hat{s}_{\tau - 1}^{(\theta)}, \hat{a}_{\tau - 1}^{(*)})
\end{equation}

\begin{equation}
    \label{eq:p_nu}
    (\mu_{\nu}(o_{\tau}), \log{(\Sigma_{\nu}(o_{\tau}))}) = p_{\nu}(o_{\tau} | \hat{s}_{\tau}^{(\theta)})
\end{equation}

\begin{equation}
    \label{eq:q_xi}
    \{\text{Prob}(a_{\tau}) \ \forall a_{\tau} \in \mathbb{A} \ | \ \mu_{\theta}(s_{\tau}), \Sigma_{\theta}(s_{\tau})\} = q_{\xi}(a_{\tau} \ | \ \mu_{\theta}(s_{\tau}), \Sigma_{\theta}(s_{\tau}))
\end{equation}

\begin{equation}
    \label{eq:efe_est}
    \tilde{G}_{\tau} = \{\tilde{g}(a_{\tau}) \ \forall a_{\tau} \in \mathbb{A} \ | \ \mu_{\theta}(s_{\tau}), \Sigma_{\theta}(s_{\tau})\} = f_{\psi}(\mu_{\theta}(s_{\tau}), \Sigma_{\theta}(s_{\tau}))
\end{equation}

\begin{equation}
    \label{eq:var_action_post}
    \{\text{Prob}(a_{\tau}) \ \forall a_{\tau} \in \mathbb{A} \ | \ \tilde{g}(a_{\tau}) \in \tilde{G}_{\tau}\} = \sigma(-\gamma_t \cdot \tilde{G}_{\tau}) = p_{\lambda}(a_{\tau} \ | \ \tilde{G}_{\tau})
\end{equation}

Where the following values are sampled from \ref{eq:q_theta} by means of the ``Reparameterization Trick'':

% Sampled values via reparameterization trick:
\begin{equation}
    \label{eq:q_phi_prev_sample}
    \hat{s}_{\tau - 1}^{(\theta)} = \mu_{\theta}(s_{\tau - 1}) + \epsilon \odot \Sigma_{\theta}(s_{\tau - 1})
\end{equation}

\begin{equation}
    \label{eq:q_phi_curr_sample}
    \hat{s}_{\tau}^{(\theta)} = \mu_{\theta}(s_{\tau}) + \epsilon \odot \Sigma_{\theta}(s_{\tau})
\end{equation}

Note: $f_{\psi}$ is a neural network that predicts the Expected Free Energy of future actions, given the multivariate Gaussian statistics over the current state. See \cite{DEEP-AIF-For-POMDPs} for details on this network. We can now obtain an approximation to the densities in \ref{eq:true_VFE} by instantiating a diagonal, multivariate Gaussian distribution for each respective prediction in the above. That is:

\begin{equation}
    \begin{aligned}
        \label{eq:approx_Gauss}
        Q(s_{\tau}) &\approx \mathcal{N}_{\theta}(\mu_{\theta}(s_{\tau}), \Sigma_{\theta}(s_{\tau})) \\
        P(s_{\tau} \ | \ s_{\tau - 1}, a_{\tau - 1}) &\approx \mathcal{N}_{\phi}(\mu_{\phi}(s_{\tau}), \Sigma_{\phi}(s_{\tau})) \\
        P(o_{\tau} \ | \ s_{\tau}) &\approx \mathcal{N}_{\nu}(\mu_{\nu}(o_{\tau}), \Sigma_{\nu}(o_{\tau})) \\
    \end{aligned}
\end{equation}

Similarly for the Categorical distributions:

\begin{equation}
    \begin{aligned}
        \label{eq:approx_Cat}
        Q(a_{\tau} \ | \ s_{\tau}) &\approx q_{\xi}(a_{\tau} \ | \ \mu_{\theta}(s_{\tau}), \Sigma_{\theta}(s_{\tau})) = \mathcal{C}_{\xi}(a_{\tau}) \\
        P(a_{\tau} \ | \ s_{\tau}) &\approx p_{\lambda}(a_{\tau} \ | \ \tilde{G}_{\tau}) = \mathcal{C}_{\lambda}(a_{\tau})
    \end{aligned}
\end{equation}

Thus, equation \ref{eq:true_VFE} becomes:

\begin{equation}
    \begin{aligned}
        \label{eq:approx_VFE}
        -F_{\tau} &= -\mathbb{E}_{\mathcal{N}_{\theta}}[\log{(\mathcal{N}_{\nu})}] \\
        &+ \text{D}_{\text{KL}}[\mathcal{N}_{\theta} \ \| \ \mathcal{N}_{\phi}] \\
        &+ \text{D}_{\text{KL}}[\mathcal{C}_{\xi}(a_{\tau}) \ \| \ \mathcal{C}_{\lambda}(a_{\tau})]
    \end{aligned}
\end{equation}


\section{Training Phase}
\label{sec:training_phase}
To train these networks, I follow the procedure laid out in \cite{learn_gen_ssm_aif}, described here.

I first obtain a dataset: $\mathcal{D}$ of $N$ observation-action pairs solicited by means of a random policy:

\begin{equation}
    \label{eq:data_D}
    \mathcal{D} = \{(o_{\tau}, a_{\tau})\} \ | \ \tau \in [0, N] \subset \mathbb{Z} \}
\end{equation}

Hence the ``starred'' terms (with superscript *) in \ref{eq:q_theta} and \ref{eq:p_phi} become the following:

\begin{equation}
    \begin{aligned}
        \label{eq:training_notation}
        \hat{a}_{\tau - 1}^{(*)} &= \hat{a}_{\tau - 1}^{(\mathcal{D})} \\
        \hat{o}_{\tau}^{(*)} &= \hat{o}_{\tau}^{(\mathcal{D})}
    \end{aligned}
\end{equation}

to indicate that they are sampled from the training dataset $\mathcal{D}$. The networks are then trained via SGD on mini-batches from $\mathcal{D}$ with respect to equation \ref{eq:approx_VFE}. Figure $3$ in \cite{learn_gen_ssm_aif} depicts this process very nicely. 

\section{Testing Phase}
\label{sec:testing_phase}
Once trained, these networks can then be used to perform Active Inference and planning. In the Testing phase, actions are now sampled from the action posterior: $q_{\xi}(a_{\tau} \ | \ \mu_{\theta}(s_{\tau}), \Sigma_{\theta}(s_{\tau}))$, instead of randomly:

\begin{equation}
    \begin{aligned}
        \label{eq:testing_notation}
        \hat{a}_{\tau - 1}^{(*)} &= \hat{a}_{\tau - 1}^{(\xi)} \\
        \hat{o}_{\tau}^{(*)} &= \hat{o}_{\tau}^{(\mathcal{D})}
    \end{aligned}
\end{equation}

As an aside: for any possible counter-factual \textit{planning} procedures such as a tree search over future states/observations, the learned observation model: $p_{\nu}(o_{\tau} | \hat{s}_{\tau}^{(\theta)})$ can be used to generate plausible observations from hidden state estimates. Thus, I would use:

\begin{equation}
    \label{eq:future_obs}
    \hat{o}_{\tau}^{(*)} = \hat{o}_{\tau}^{(\nu)}
\end{equation}

in the course of a ``planning as inference'' procedure. I do not consider any counter-factual \textit{planning} procedures here \textemdash \ although that is my ultimate goal \textemdash \ and so I leave this for another time. 

\section{References}

\printbibliography


\end{document}